\documentclass[a4paper]{article}
\usepackage[T1]{fontenc}
\usepackage[utf8]{inputenc}
\begin{document}
\begin{tabular}{|l|p{10cm}|l|l|}
\hline
Nr& Lýsing& Forgangur& Uppl\\
\hline
1& Notandi getur skráð sig inn & A &\\
\hline
2& Kerfið hefur texta viðmót & A &\\
\hline
3& Yfirmaður á að geta skráð nýja-  listað og breytt upplýsingum um starfsmenn & A &\\
\hline
4& Yfirmaður á að geta skráð nýja- listað og breytt upplýsingum um fasteignir & A &\\
\hline
5& Yfirmaður á að geta stofnað nýja og/eða breytt verkbeiðni fyrir sínar fasteignir & A &\\
\hline
6& Yfirmaður á að geta samþykkt tilbúnar verkskýrslur og þannig lokað sínum verkbeiðnum & A &\\
\hline
7& Starfsmenn þurfa að geta skráð verkskýrslu á opna verkbeiðni & A &\\
\hline
8& Starfsmenn þurfa að geta merkt verkbeiði sem tilbúna fyrir lokun af yfirmanni & A &\\
\hline
9& Það þarf að vera hægt að lista og leita eftir starfsmönnum & A &\\
\hline
10& Það á að vera hægt að lista og leita eftir fasteignum & A &\\
\hline
11& Notandi getur fengið lista yfir starfsmenn & A &\\
\hline
12& Notandi getur fengið lista yfir fasteignir & A &\\
\hline
13& Notandi getur leitað að fasteign eftir auðkenni & A &\\
\hline
14& Notandi getur leitað að verkbeiðni eftir númeri & A &\\
\hline
15& Notandi getur fengið lista yfir verkbeiðnir og skýrslur & A &\\
\hline
16& Notandi getur fengið allar verkbeiðnir og skýrslur fyrir ákveðna fasteign & A &\\
\hline
17& Notandi getur fengið allar verkbeiðnir sem gerðar eru af ákveðnum starfsmanni & A &\\
\hline
18& Þarf að vera skrá fyrir áfangastaði sem NaN flýgur á & A &\\
\hline
19& Yfirmaður þarf að getað skráð nýja og breytt upplýsingum um verktaka & B &\\
\hline
20& Stafsmenn geta fengið upplýsingar um verktaka & B &\\
\hline
21& Þarf að vera hægt að vísa í verktaka í verkbeiðnum & B &\\
\hline
22& Verkskýrslur innihalda þóknun verktaka & B &\\
\hline
23& Yfirmenn geta komið fram athugasemdum um verkið þegar hann lokar beiðnum & B &\\
\hline
24& Starfsmenn getað kallað í lista með viðhaldsverkefnum á þeirra staðsetningu og dagsetningu & B &\\
\hline
25& Verkbeiðnir hafa endurtektar valmöguleika sem bíður upp á að hafa verkefni reglubundin & B &\\
\hline
26& Verkbeiðnir hafa þriggja flokka forgangsröðun & B &\\
\hline
27& Yfirmenn sjá um að setja forgang á verkefni og geta breytt honum á líftíma verkefnis  & B &\\
\hline
28& Hægt að fá lista yfir allt viðhald á ákveðinni fasteign á ákveðnu tímabili  & B &\\
\hline
29& Hægt að fá yfirlit yfir öll verk ákveðins verktaka á ákveðnu tímabili  & B &\\
\hline
30& Hægt að fá yfirlit yfir öll verk sem ákveðin starfsmaður hefur sinnt á ákveðnu tímabili  & B &\\
\hline
31& Notandi getur leitað eftir stikkorðum í skýrslum og verkbeiðnum með valkvæðu case-sensitivity& C &\\
\hline
32& NOTANDI fær viðvörun áður en verktaki með slæma umsögn úr seinasta verkefni er ráðinn & C &\\
\hline
33& Notandi getur fengið prentvæn yfirlit yfir útistandandi verkbeiðnir & C &\\
\hline
34& Notandi getur fengið prentvænar útgáfur af gagnlegum skiltum & C &\\
\hline
35& \textbackslash{}textit\{Notandi getur fengið veðurspá fyrir daginn\} &C &\\
\hline
36& \textbackslash{}textif\{Notandi getur skráð sig úr kerfi án þess að slökkva á kerfi\} & C &\\
\hline
\end{tabular}
\end{document}
